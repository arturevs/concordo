\chapter{README}
\hypertarget{md_README}{}\label{md_README}\index{README@{README}}
{\bfseries{Esse é o projeto Concordo da disciplina Linguagem de Programação 1 ministrada pelo professor Sidemar na UFRN para o curso de BTI.}}

Autor\+: {\itshape Artur Revorêdo Pinto}

Horário da turma\+: {\itshape M56}

{\bfseries{Para compilar o código basta rodar os seguintes comandos}}


\begin{DoxyItemize}
\item cmake -\/B build
\item cmake --build build
\end{DoxyItemize}

{\bfseries{Para rodar o programa basta utilizar o comando}}


\begin{DoxyItemize}
\item ./build/concordo
\end{DoxyItemize}

Nesse projeto existem alguns comandos possíveis, caso você não esteja logado os comandos são\+:


\begin{DoxyItemize}
\item create-\/user email senha nome

Nesse comando email nem senha podem conter espaços, o nome no entanto pode.
\item login email senha

Entra com um usuário no sistema.
\item quit

Sai do programa.
\end{DoxyItemize}

Caso o login seja efetuado então outros comandos ficarão disponíveis para o usuário, esses são.


\begin{DoxyItemize}
\item disconnect

Desconecta o usuário atual do sistema.
\item create-\/server nome

Em que nome é o nome do servidor.
\item set-\/server-\/desc nome descrição

Nesse o nome é o nome do servidor e descrição é a descrição que você deseja inserir no servidor.
\item set-\/server-\/invite-\/code nome invite-\/code

Nesse o nome é o nome do servidor o qual o invite code será mudado e invite code é um código sem espaços para ser inserido no servidor
\item list-\/servers

Apenas lista os servidores existentes
\item remove-\/server

Deleta um servidor do sistema.
\item join-\/server nome invite code

Nesse o usuário tentará ser adicionado no servidor de nome \"{}nome\"{} e digitará o código de convite.
\item enter-\/server nome

Nesse o usuário entra em um servidor do qual já é membro.
\item leave-\/server

Nesse o usuário sai do servidor, não deixando de ser um participante mas apenas para navegar por outros servidores.
\item create-\/channel nome

Nesse o usuário cria um canal no servidor em que está no momento com o nome dado no comando.
\item enter-\/channel nome

Nesse o usuário entrará no canal de nome informado no comando caso exista no servidor atual.
\item leave-\/channel

Nesse o usuário sairá do canal atual.
\item send-\/message mensagem

Nesse o usuário envia uma mensagem nova no canal atual se estiver em um.
\item list-\/messages

Nesse o usuário lista as mensagens do canal atual. 
\end{DoxyItemize}